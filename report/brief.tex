%-------------------------------------------------------------------------------
\documentclass{article}

%-------------------------------------------------------------------------------
% Packages
\usepackage{amsmath}
\usepackage[portuguese]{babel}
\usepackage{othelloboard}

%-------------------------------------------------------------------------------
% User-commands
\newcommand{\todo}[1]{{\color{red}{#1}}}

%-------------------------------------------------------------------------------
% Project configs
\title{Relatório de I.A.: Gomoku, Parte 1}
\author{Cauê Baasch de Souza \\
        João Paulo Taylor Ienczak Zanette}
\date{\today}

%-------------------------------------------------------------------------------
\begin{document}
    \maketitle{}

    \section{Modelo de estados e do jogo}

    O projeto foi modelado como:

    \begin{description}
        \item [Estados:] Cada estado é uma matriz (implementada como um
            \textit{array} bi-dimensional) cujos elementos representam as casas
            do tabuleiro, indicando se há uma pedra nela e a que jogador ela
            pertence.

        \item [O jogo:] Uma estrutura que contém o estado atual do tabuleiro,
            os dois jogadores (dados como instâncias de qualquer tipo que
            obedeça uma interface padrão \texttt{Player}), um indicador
            que marca de quem é a vez, e o número total de jogadas até o
            momento.
    \end{description}

    \section{Heurística}

    Para definir a função, são definidos os seguintes cenários:

    \begin{description}
        \item [Vitória:] Um cenário com uma quíntupla contígua.

        \item [Vitória iminente:] Quando a vitória é garantida apenas colocando
        uma única peça no tabuleiro. Isso ocorre quando há:
            \begin{itemize}
                \item Uma quádrupla (não necessariamente contígua) com espaço
                    para virar uma quíntupla (Figura~\ref{win-four}).
                \item Duas triplas (não necessariamente contíguas) com um
                    espaço compartilhado (Figura~\ref{win-shared-triple}).
                \item Dois espaços separados apenas por uma permutação de 4
                    elementos $\langle X, X, X, E \rangle$, em que X é uma peça
                    do jogador atual (Figura~\ref{win-four-permut}).
            \end{itemize}

        \item [Possível vitória:] Um cenário de vitória iminente com uma peça
        do jogador substituída por um espaço vazio.
    \end{description}

    Sendo assim, a função adotada está descrita na Equação~\ref{heuristic-function}:

    \begin{equation}
        H(s) = \begin{cases}
            \infty, & \mbox{Se há pelo menos um cenário de vitória}
                      \vspace{1em} \nonumber \\
            c_i * Q_i + c_p * Q_p(s) + c_j * Q_j(s) + \sum_{e\in E(s)} c_e * {Q_p}_e(s), & \mbox{Em qualquer outro caso} \nonumber \\
        \end{cases}
        \label{heuristic-function}
    \end{equation}

    Em que:

    \begin{itemize}
        \item $s$ é um estado do jogo;
        \item $c_x$ é uma constante relacionada à categoria de $x$;
        \item $Q_i(s)$ é a quantidade de cenários de vitória iminente distintos
        presentes em $s$;
        \item $Q_p(s)$ é a quantidade de cenários de possível vitória distintos
        presentes em $s$;
        \item $Q_j(s)$ é a quantidade de jogadas totais que foram feitas até o
        jogo chegar em $s$;
        \item $E(s)$ é o conjunto de casas em branco (espaços) presentes em $s$;
        \item ${Q_p}_e(s)$ é a quantidade de cenários de possível vitória
        distintos em $s$ que compartilham o mesmo espaço $e$.
    \end{itemize}

    Também é calculada a heurística do ponto de vista do oponente,
    e subtraída do valor calculado para si.

    \section{Estruturas adicionais}

    Faremos uso de uma árvore cujos vértices representam os estados do jogo,
    com um marcador que indica a qual jogador está associado o nível e os
    valores de alfa, beta, e da heurística. Os filhos de um vértice serão os
    estados do jogo aos quais é possível chegar em uma única jogada.

    \section{Otimizações planejadas}

    \begin{itemize}
        \item Podas alfa-beta;
        \item Aprofundamento progressivo;
        \item \textit{Multi-threading};
        \item Tempo limite para cálculo de uma jogada.
    \end{itemize}

    \begin{figure}[ht]
        \centering
        \caption{Exemplo de quádrupla contígua. No exemplo, o
        jogador garante sua vitória quando posicionar uma peça
        em $(1, a)$ ou $(1, f)$.\label{win-four}}
        \begin{othelloboard}{1}
            \dotmarkings{}
            \drawboardfromstring{-XX-XX--}
        \end{othelloboard}
    \end{figure}

    \begin{figure}
        \centering
        \caption{Exemplo de permutação de $\langle X, X, X, E
        \rangle$. No exemplo, posicionando uma peça em $(1, c)$
        garante vitória no turno seguinte, sem chances de o
        oponente escapar.\label{win-four-permut}}
        \begin{othelloboard}{1}
            \dotmarkings{}
            \drawboardfromstring{%
                -X-XX---%
            }
        \end{othelloboard}
    \end{figure}

    \begin{figure}
        \centering
        \caption{Exemplo de duas triplas não-contíguas. No
        exemplo, posicionando uma peça em $(1, a)$ garante
        vitória no turno seguinte, sem chances de o oponente
        escapar.\label{win-shared-triple}}
        \begin{othelloboard}{1}
            \dotmarkings{}
            \drawboardfromstring{%
                -X-XX---%
                --------%
                X-------%
                X-------%
                X-------%
            }
        \end{othelloboard}
    \end{figure}
\end{document}
